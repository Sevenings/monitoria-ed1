\documentclass[a4paper]{article}

\usepackage[brazil]{babel}
\usepackage{ulem, hyperref, verbatim, tcolorbox, indentfirst}

\hypersetup{
	colorlinks=true,
	urlcolor=blue
}
\urlstyle{same}

\author{Lucas Vinícius de Lima Assis\\ \small{(62) 9 9973-7345}}

\title{Guia de Estudos \\ Práticas com EDs}

\begin{document}
    \maketitle
    \begin{center}
        \textbf{Considerações iniciais}
    \end{center}

    \textit{``Este é um pequeno guia sobre prática com algumas estruturas de dados.
        Como o último Guia de Estudos foi sobre os \emph{pré-requisitos} para se
    trabalhar com estruturas de dados (Ponteiros, malloc e structs), neste guia
    eu trago algumas plataformas que vocês podem usar para encontrar questões,
    sempre procurando dar a vocês os materiais para terem independência no seu
    aprendizado.}   

    \textit{Já adianto, recomendo que implementem, no mínimo uma vez, uma lista
    encadeada, duplamente encadeada, circular, tratem como uma pilha, e depois
    como uma fila, e façam no mínimo as operações básicas:
    adicionar no começo, no final, printar todos os elementos, busca de
    elementos. Essas são operações básicas.} 

    \textit{Parece muito, mas vocês vão ver que é tranquilo, exemplo: entre uma
    pilha e uma fila, a única diferença é a inserção e retirada, a busca é igual
    à da lista encadeada, não precisa repetir.}

    \textit{Fazer as operações ao menos uma vez ativa teu cérebro para essa nova maneira
    de pensar usando ponteiros."}
    \begin{flushright}
        \textit{-O Monitor}
    \end{flushright}

    \section*{Beecrowd}
    O \href{https://www.beecrowd.com.br/}{Beecrowd} é um site de
    \emph{programação competitiva}. Ele contém um banco de questões enorme. A
    melhor parte é que o próprio site verifica se sua resposta está correta,
    passando seu código por centenas de casos de testes.
    
    Sugiro então que criem uma conta \textit{(já aviso que é chatinho, mas tenham
    animo)} e vão na aba \emph{Problems}$\to$\emph{Structs} que vão encontrar
    problemas sobre \emph{estruturas de dados}. Aqui estão 4 problemas que
    separei:

    \begin{itemize}
        \item \href{https://www.beecrowd.com.br/judge/en/problems/view/1068}{Parenthesis Balance I}
        \item \href{https://www.beecrowd.com.br/judge/en/problems/view/1069}{Diamonds
            and Sand}
        \item
            \href{https://www.beecrowd.com.br/judge/en/problems/view/1077}{Infix
            Postfix}
        \item
            \href{https://www.beecrowd.com.br/judge/en/problems/view/1083}{LEXSIM
            - Sintatic and Lexical Avaliator}
    \end{itemize}

    Uma vantagem desse tipo de questão é que as perguntas são abertas, vocês
    podem resolver de diversas maneiras, e vocês que devem avaliar qual
    estrutura de dados utilizar, ou até mesmo, se realmente precisam utilizar
    alguma.

    Pessoalmente, quando eu resolvo os exercícios no beecrowd, muitas vezes eu
    opto por implementações mais simples das estruturas de dados, porque eles me
    dão um problema pontual, e não um software que pode expandir e, então,
    requer o código mais genérico possível. Assim, recomendo que se resolverem
    de uma maneira mais, digamos, \textit{`sagaz'}, se desafiem a resolver com uma
    implementação clássica, e ver a diferença de performance entre as duas
    soluções.
    % \newpage{}
    \section*{Sugestões Materiais Externos}
    Sugiro que procurem também materiais pela internet mais \emph{específicos} sobre o
    conteúdo. O Instituto de Computação (IC) da Unicamp possui diversos
    materiais sobre o assunto, estudei por lá a parte de listas encadeadas.  A
    seguinte questão foi retirada do
    \href{https://www.ic.unicamp.br/~thelma/gradu/MC102/Turma-2008S2/Listas-exercicios/Lista6_exerc.pdf}{Material
    da professora Thelma Chiossi}. 

    \paragraph{Subdividir uma \textit{Linked List}\\}
    Considere a implementação:

    \begin{tcolorbox}
        \begin{verbatim}
typedef struct lista{
    int info;
    struct lista *prox;
}Lista;
        \end{verbatim}
    \end{tcolorbox}
    Escreva uma função que receba, como parâmetros, uma lista e um número inteiro n,
e divida a lista em duas, de forma que a segunda lista comece no primeiro nó logo
após a ocorrência de n na lista original. A função deve retornar um ponteiro para a
segunda subdivisão da lista original, enquanto L (cabeça da lista original) deve
continuar apontando para o primeiro elemento da primeira subdivisão da lista.

    \textit{ Depois deem uma olhada e tentem fazer as outras questões, vai que você
    encontra algo que parece difícil lá.}

    \section*{Dica: Aprendam a fazer bibliotecas}
    Sendo bem direto, é muito útil ter uma biblioteca com uma implementação de
    uma lista encadeada, outra com uma lista duplamente encadeada, outra com uma
    pilha, uma pilha com implementação de vetor, com implementação de lista
    encadeada, e por aí vai\ldots Vocês vão precisar tantas vezes dessas mesmas
    implementações, que vale a pena guardar tudo num arquivo, e quando precisar,
    só dar um \verb|#include "BolsaDaHermione.h"|. Sugiro esse
    \href{https://drive.google.com/file/d/1GKyZKR66TfSpYVjgxOq4uDup1GitjZtC/view?usp=sharing}{material
    do IC da Unicamp}.
\end{document}

