\documentclass[a4paper]{article}

\usepackage[brazil]{babel}
\usepackage{ulem}
\usepackage{hyperref}
\usepackage{verbatim}
\usepackage{tcolorbox}

\hypersetup{
	colorlinks=true,
	urlcolor=blue
}
\urlstyle{same}

% \href{https://github.com/Sevenings/Projeto-Final-Lia}{Repositório do Projeto}

\author{Lucas Vinícius de Lima Assis}

\title{Guia de Estudos \\ Listas Dinâmicas}

\begin{document}
    \maketitle
    

    
    \textit{Este é um pequeno guia sobre as bases a se estudar para começar a
    implementar estruturas de dados em C. Para se implementar a mais básica, a
    Lista Encadeada, é necessário saber fazer uma \uline{Alocação Dinâmica}, criar
    \uline{Structs} e usar \uline{Ponteiros}.}

    \textit{Este é apenas um guia de estudos, e todo o conhecimento teórico já está
    disponível em bons livros. Recomendo que usem o}
    \href{https://www.inf.ufpr.br/lesoliveira/download/c-completo-total.pdf}{C
    Completo Total}. \textit{Vão encontrar até mais coisas do que buscavam
        inicialmente.}

    \textit{Conseguindo fazer esses exercícios básicos creio que possuam tudo o que é
    preciso para implementar as estruturas de dados. No entanto, não se limitem
    a este mero guia.}

    \section*{Ponteiros}

    Este é um código em C que troca os valores de duas variáveis utilizando uma
    função de troca. No entanto, a função foi apagada. Utilize o conhecimento
    sobre ponteiros e seus operadores \& e * para reescrever uma função de troca
    funcional. 

    \begin{tcolorbox}
        \verbatiminput{programa_troca.c}
    \end{tcolorbox}

    Note que a função \verb|void troca(int* a, int* b)| recebe ponteiros como
    parâmetros. Isso é realmente necessário? Seria possível fazer uma função
    \verb|void troca(int a, int b)| que funcione? 

    \section*{Malloc e Free}

    Este código tem gambiarra! Embora ele funcione como o desejado, perceba que
    estamos fazendo uma declaração de um vetor estático de maneira "dinâmica".
    Refatore o código utilizando \verb|malloc| e \verb|free| para que a alocação
    dinâmica seja realizada corretamente.

    \begin{tcolorbox}
        \verbatiminput{vetores_est.c}
    \end{tcolorbox}


    \newpage

    \section*{Structs}

    Recentemente, comecei a programar um jogo de Digimon em C. Decidi começar
    simples e meus Digimons vão ter apenas \textit{vida} e \textit{dano}, todos
    do tipo \textit{int}. Criei uma \verb|struct| para representar os Digimons,
    e uma função \verb|NovoDigimon|. Essa função aloca
    dinamicamente o espaço de um Digimon, atribui a vida e o dano neste espaço
    de memória e me retorna o ponteiro deste Digimon alocado. Por algum motivo
    eu não salvei a implementação da struct e da função e já até esqueci como
    tinha feito. Poderia me ajudar?

    \begin{tcolorbox}
        \verbatiminput{structs_digimon.c}
    \end{tcolorbox}

\end{document}
