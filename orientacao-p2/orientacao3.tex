\documentclass[a4paper]{article}

\usepackage[brazil]{babel}
\usepackage{ulem, hyperref, verbatim, tcolorbox, indentfirst}

\hypersetup{
	colorlinks=true,
	urlcolor=blue
}
\urlstyle{same}

\author{Lucas Vinícius de Lima Assis\\ \small{(62) 9 9973-7345}}

\title{Manipulações em \\ Listas Encadeadas }

\begin{document}
    \maketitle
    \begin{center}
        \textbf{Considerações iniciais}
    \end{center}

    \textit{``Este é um pequeno guia para treinar a leitura de código e revisar
        algumas manipulações com listas encadeadas. Eu não sei as questões da
        prova, mas tenho certeza de que estar familiarizado com essas
        operações elementares ao ponto de ler um código e entender o que ele faz
        com a lista, \emph{faz parte} de um bom preparo. Enfase para o \emph{faz
        parte}, porque ainda existe muito mais a ser explorado que eu não consigo
        trazer aqui neste pequeno documento. Sem mais delongas, bons estudos,
        e qualquer dúvida me chamem no zap, eu tenho o gabarito."}
    \begin{flushright}
        \textit{-O Monitor}
    \end{flushright}

    \section*{Identificar operações}

    Eu andei programando um pouco esses dias e implementei uma lista
    simplesmente encadeada e algumas operações para mexer com ela. Vou colocar
    alguns dos códigos das funções para vocês identificarem o que cada função
    faz. Já vou dizendo que há algumas que não funcionam para todos os casos,
    fiquem atentos a isso também e identifiquem as que não funcionam para
    qualquer caso. E o principal: \textbf{todas possuem um
    significado, não são operações aleatórias}.

    Para todas as funções, eu utilizei essa declaração de Lista/Nó. A lista no
    caso é um ponteiro para o primeiro Nó.

\begin{tcolorbox}
    \begin{verbatim}
// Declaração da Lista
typedef struct No {
    int info;
    struct No *prox;
} No;
    \end{verbatim}
\end{tcolorbox}


\begin{tcolorbox}
    \begin{verbatim}
// Função A
No* funcaoA() {
    return NULL;
}

// Exemplo de uso
int main() {
    No* lista = funcaoA()
    ...
}
    \end{verbatim}
\end{tcolorbox}


Sobre a próxima função, por que eu usei \verb|No** plista| e não \verb|No* lista|?
Eu poderia fazer do segundo jeito?
\begin{tcolorbox}
    \begin{verbatim}
// Função B
void funcaoB(No** plista, int v) {    
    No* n = (No*) malloc(sizeof(No));
    n->info = v;
    n->prox = (*plista);
    (*plista) = n;
}

// Exemplo de uso
int main () {
    No *lista = funcaoA();
    funcaoB(&lista, 4);
    funcaoB(&lista, 12);
    ...
}
    \end{verbatim}
\end{tcolorbox}

As funções $C_a$ e $C_b$ pretendem fazer a mesma coisa, mas de maneiras
diferentes. Ambas funcionam? Qual a diferença entre elas? 
\begin{tcolorbox}
    \begin{verbatim}
// Função Ca
void funcaoCa(No* lista) {
    No* aux = lista;
    while (aux != NULL) {
        printf("(%d)->", aux->info);
        aux = aux->prox;
    }
    printf("NULL\n");
}
    \end{verbatim}
\end{tcolorbox}


\begin{tcolorbox}
    \begin{verbatim}
// Função Cb
void funcaoCb(No* lista) {
    while (lista != NULL) {
        printf("(%d)->", lista->info);
        lista = lista->prox;
    }
    printf("NULL\n");
}
    \end{verbatim}
\end{tcolorbox}


\begin{tcolorbox}
    \begin{verbatim}
// Função D
void funcaoD(No **plista) {
    No *aux = (*plista);
    No *l = (*plista);

    while (l != NULL) {
        l = l->prox;
        free(aux);
        aux = l;
    }
    (*plista) = NULL;
}
    \end{verbatim}
\end{tcolorbox}

\begin{tcolorbox}
    \begin{verbatim}
// Função E
void funcaoE(No **plista, int valor) {
    No *n = (No*) malloc(sizeof(No));
    n->info = valor;
    
    No *l = (*plista);
    No *p = l->prox;

    while (p != NULL && p->info < valor) {
        l = l->prox;
        p = p->prox;
    }

    n->prox = p;
    l->prox = n;
}
    \end{verbatim}
\end{tcolorbox}

\begin{tcolorbox}
    \begin{verbatim}
// Função F
void funcaoF(No* lista) {
    No *a, *b;

    int t = 1;
    while (t) {
        a = lista;
        b = a->prox;
        t = 0;
        while (b != NULL) {
            if (a->info > b->info) {
                int aux = a->info;
                a->info = b->info;
                b->info = aux;
                t = 1;
            }
            a = b;
            b = b->prox;
        }
    }
}
    \end{verbatim}
\end{tcolorbox}

\begin{tcolorbox}
    \begin{verbatim}
// Função G
void funcaoG(No **plista, int n, int l) {
    for (int i=0; i<n; i++) {
        funcaoB(plista, rand()%l);
    }
}
    \end{verbatim}
\end{tcolorbox}

\begin{tcolorbox}
    \begin{verbatim}
// Função H
No* funcaoH(No *lista, int n) {
    while (lista != NULL && lista->info != n) {
        lista = lista->prox;
    }
    return lista;
}
    \end{verbatim}
\end{tcolorbox}


\begin{tcolorbox}
    \begin{verbatim}
// Função Ia
void funcaoIa(No** plista) {
    No *a, *b, *c;
    a = (*plista);
    b = a->prox;
    c = NULL;
    while (a != NULL) {
        a->prox = c;
        c = a;
        a = b;
        if (b != NULL)
            b = b->prox;
    }
    (*plista) = c;
}
    \end{verbatim}
\end{tcolorbox}


\begin{tcolorbox}
    \begin{verbatim}
// Função Ib
void funcaoIb(No** plista, No* a, No* b, No* c) {
    a->prox = c;
    c = a;
    a = b;
    if (b != NULL)
        b = b->prox;

    if (a != NULL) {
        funcaoIb(plista, a, b, c);
    } else {
        (*plista) = c;
    }
}

// Exemplo de uso
int main() {
    ...
    No* lista = funcaoA();
    funcaoG(&lista, 20, 10);
    funcaoIb(&lista, lista, lista->prox, NULL);
}
    \end{verbatim}
\end{tcolorbox}


\end{document}

